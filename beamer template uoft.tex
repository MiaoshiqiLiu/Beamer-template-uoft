\documentclass [aspectratio=169]{beamer}
\usetheme{Boadilla}
\usepackage{textpos} % package for the positioning
\usepackage[]{graphicx}
\usepackage{graphicx}
\usepackage{float}
\usepackage{hyperref}
\usepackage{caption}
\usepackage{subcaption}
\usepackage{algorithm,algpseudocode}
\usepackage[export]{adjustbox}
\usepackage{tikz}
\usetikzlibrary{positioning}
\usetikzlibrary{arrows, shapes, decorations, automata, backgrounds, fit, petri, calc}
\setbeamertemplate{itemize items}[circle]
\setbeamertemplate{enumerate items}[circle]

\newcommand*{\logofont}{\fontfamily{phv}\selectfont}
\definecolor{uoftblue}{RGB}{6,41,88} % official blue color for uoft

\vspace{1in}
\title[]{Simultaneous nonparametric inference under \\ complex temporal dynamics}
\author[]{Miaoshiqi (Shiki) Liu}
\institute[]{University of Toronto}
\date{\today}

% set color
\setbeamercolor{title in head/foot}{bg=white}
\setbeamercolor{author in head/foot}{bg=white}
\setbeamercolor{date in head/foot}{fg=uoftblue}
\setbeamercolor{date in head/foot}{bg=white}
\setbeamercolor{title}{fg=uoftblue}
\setbeamerfont{title}{series=\bfseries}
\setbeamercolor{frametitle}{fg=uoftblue}
\setbeamerfont{frametitle}{series=\bfseries}
\setbeamercolor*{item}{fg=uoftblue}
\setbeamercolor{block title}{bg=uoftblue}
\setbeamercolor{block title}{fg=white}
\setbeamercolor{block body}{bg=uoftblue!5!white}

% set logo at non-title pages
\logo{\includegraphics[height=0.8cm]{logo_uoft.png}\vspace*{-.055\paperheight}\hspace*{.85\paperwidth}}

% set margin
\setbeamersize{text margin left=10mm,text margin right=10mm}

\begin{document}
{
\setbeamertemplate{logo}{}
\begin{frame}
    \vspace{0.5in}
    \titlepage
    \begin{textblock*}{4cm}(0.5cm,-7.5cm)
        \includegraphics[width=4cm]{logo_uoft.png}
    \end{textblock*}
    \begin{textblock*}{8cm}(5.0cm,-7cm)
        \huge \color{uoftblue}{$\Bigr\rvert$ \hspace{0.15cm} \textbf{\logofont Statistical Sciences}}
    \end{textblock*}
\end{frame}
}

\begin{frame}{Outline}
    \begin{itemize}
    	\item Introduction
        \item Setting
        \item Main Theorems
        \item Applications to Testings
        \item Simulation Results
        \item Real Data Analysis
    \end{itemize}
\end{frame}

\begin{frame}{Introduction}
\textbf{Model}
\begin{block}{Varying Coefficient Model}
\begin{equation*}
    y_i=\mathbf{x}_i^{\top}\boldsymbol{\beta}_i+e_i, \quad i=1,2,\cdots,n,
\end{equation*}
\end{block}
where $\{\mathbf{x}_i=(x_{i,1},\cdots,x_{i,p})^\top\}$ is the $p$-dimensional covariate (or predictor) process,  $\{e_i=(e_{i,1},\cdots,{e_{i,p}})^\top\}$ is the $p$-dimensional error process.

    \begin{itemize}
        \item 
        \item 
        \item  
        \item 
    \end{itemize}
\end{frame}


\begin{frame}{Setting}
    \begin{itemize}
        \item 
        \item 
        \item 
    \end{itemize}
\end{frame}

\begin{frame}{Main Theorems}
    \begin{itemize}
        \item 
    \end{itemize}
\end{frame}


\begin{frame}{Applications to Testings}
	\begin{itemize}
        \item 
    \end{itemize}
\end{frame}


\begin{frame}{Simulation Results}
Three types of tests are considered in the simulation experiment:
\begin{itemize}
    \item Exact Function Test: $\mathbf{C}\boldsymbol{\beta}(t) = f(t), \quad t \in [0,1]$
    \item Lack-of-fit Test: $\Lambda_{\mathbf{C}}(t) = f(t, \{\Lambda(t_i)\}_{i \in \mathcal{J}}), \quad t \in [0,1]$
    \item Qualitative Test: $\Lambda_{\mathbf{C}}(t) \in \mathcal{N}_0$
\end{itemize}
\end{frame}


\begin{frame}{Real Data Analysis}
    \begin{itemize}
    \item
    \end{itemize}
\end{frame}



\begin{frame}{Algorithm}

    \begin{algorithm}[H]
        \caption{Dynamic generative model}
        \begin{algorithmic}
        \State Initialize parameters $\theta, \phi$ 
        \Repeat
        \State Get random minibatch datapoints $\mathbf{x}, \mathbf{u}$
        \State Get Monte Carlo samples $\mathbf{z}^* $ from distribution $q_{\phi}(\mathbf{z}|\mathbf{x}, \mathbf{u})$
        \State Evaluate $\mathbb{E}_{\mathbf{z} \sim q_{\phi}} [\log p_{\theta}(\mathbf{x}|\mathbf{z}, \mathbf{u})]$ using $\mathbf{z}^* $
        \State Update parameters using gradients $\nabla_{\theta,\phi}\mathcal{L}$ (e.g. SGD)
        \Until {convergence of parameters $\theta,\phi$}
        \\ \Return $\theta, \phi$
        \end{algorithmic}
        \label{algorithm}
    \end{algorithm}
	
\end{frame}

\begin{frame}{Mathematical Environment Blocks}
    \begin{definition} 
        This is a definition.
    \end{definition}
    
    \begin{theorem} 
        This is a theorem. 
    \end{theorem}
    
    \begin{lemma} 
        This is a proof idea.
    \end{lemma}
\end{frame}


\end{document}
